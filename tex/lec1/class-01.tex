\documentclass[12pt]{report}	
\usepackage{scribe_MG}

%\documentclass[12pt,a4paper]{article}
\usepackage[margin=2cm]{geometry}
\usepackage[utf8]{inputenc}
\usepackage{scribe_MG}
\usepackage{amsmath}
\usepackage{graphicx}
\usepackage[]{algorithm2e}
\usepackage{hyperref}
\usepackage{color}
\usepackage{bbm}

% Style
%\theoremstyle{definition}
\newtheorem{example}{Example}[section]
%\newenvironment{example}[1]{\par \noindent\textbf{Example: #1}\\text more text}{}

% Formatting
\oddsidemargin .00in
\evensidemargin .00in
\marginparwidth 0.07 true in
\addtolength{\headsep}{0.25in}
\addtolength{\voffset}{-0.5in}
\textheight 8.5 true in
\textwidth 6.5 true in
\clubpenalty=10000

% ----------------------------------------------------------------------
%  Standard lemma, theorem, and corollary environments.
% ----------------------------------------------------------------------

\newcommand{\BlackBox}{\rule{1.5ex}{1.5ex}}  % end of proof
\newenvironment{proof}{\par\noindent{\bf D\'emonstration\ }}{\hfill\BlackBox\\[2mm]}
\newtheorem{lemma}{Lemme}[chapter]
\newtheorem{theorem}[lemma]{Theorem}
\newtheorem{corollary}[lemma]{Corollary}
\newtheorem{proposition}[lemma]{Proposition}
\newtheorem{definition}[lemma]{Definition}

% text abbrevs
\newcommand{\cf}{{\it cf.}}
\newcommand{\eg}{{\it e.g.}}
\newcommand{\ie}{{\it i.e.}}
%\newcommand{\etc}{{\it etc.}}
\newcommand{\ones}{\mathbf 1}
% std math stuff
\newcommand{\reals}{{\mbox{\bf R}}}
\newcommand{\integers}{{\mbox{\bf Z}}}
\newcommand{\eqbydef}{\mathrel{\stackrel{\Delta}{=}}}
\newcommand{\complex}{{\mbox{\bf C}}}
\newcommand{\symm}{{\mbox{\bf S}}}  % symmetric matrices


% Shortcuts
\newcommand{\X}{\mathcal{X}}
\newcommand{\Y}{\mathcal{Y}}
\let\ksi\xi
\renewcommand{\xi}{x^{(i)}}
\newcommand{\yi}{y^{(i)}}
\newcommand{\xit}{x^{(i_t)}}
\newcommand{\yit}{y^{(i_t)}}
\newcommand{\argmax}{\arg\!\max}


\newcommand{\E}{\mathbb{E}}
\newcommand{\Var}{\mathrm{Var}}
\newcommand{\Cov}{\mathrm{Cov}}
\newcommand\given[1][]{\:#1\vert\:}
%\DeclarePairedDelimiter{\abs}{\lvert}{\rvert}
%\DeclarePairedDelimiter{\norm}{\lVert}{\rVert}


\begin{document}

\course{IFT 6132 -- } 		
\coursetitle{Advanced Structured Prediction}	
\semester{Winter 2019}		
\lecturer{Simon Lacoste-Julien}	
\scribe{Frederic Boileau}		
\lecturenumber{1}		
\lecturedate{January 8, Universit\'e de Montr\'eal}	
\maketitle

\section{Structured Prediction Basis and Setup}
Given a training dataset
\begin{equation}
    D \eqbydef (\xi, \yi)_{i=1}^n \qquad \xi \in \X \quad \yi \in \Y \quad i=1,\ldots,n
\end{equation}
we want to learn a \emph{prediction mapping} 
$h_\omega \, : \, \X \rightarrow \Y$ where $\omega$ is a parameter for the
mapping.\\
Moreover we want that mapping to have \emph{low generalization error}:
\begin{equation}
    L(\omega;P) \eqbydef \E_{(x,y) \sim P} [ l(y, h_\omega \given x)]
    \label{eq:vrisk}
\end{equation}
In the above equation $L$ is the generalization error\footnote{also known as the
Vapnik risk in ML literature} which takes as input the parameter 
of the prediction mapping and $P$ which is an (arbitrary?) `test distribution'.
The function $l(\cdot,\cdot)$ on the RHS is the \emph{structured error
function} which evaluates how bad the different mistakes are.\\

\noindent\textbf{Note:} See lectures 4 and 5 of IFT6269 (Probabilistic
Graphical Models) for a review of statistical decision theory%
\footnote{\url{http://www.iro.umontreal.ca/~slacoste/teaching/ift6269/A18/}}\\

\noindent Two aspects characterize structured prediction in contrast to traditional
classification:
\begin{enumerate}
\item the set of labels $\Y$ is usually \textbf{exponential in size}.
\item We need an \textbf{encoding function} to
    represent the labels $y=(y_1,\ldots,y_d)$. The ``pieces'' of $y$ have
    constraints.
\item The prediction task is characterized by a \textbf{structured error/loss
    function} $\ell(y,y')$. In other words we do not automatically get
    an obvious metric for the loss (such as mean squared error in simple
    linear regression).
\end{enumerate}


We need an error function on complex labels because not all mistakes are equal.
For instance in machine translation making a mistake on a single word is not as
bad as getting the whole sentence wrong. In the case of scene labelling it is a
much smaller mistake to label a car as a truck than say, a pedestrian as the
road. Note that error functions do not necessarily have to be \emph{metrics}
as rigorously defined in analysis.

\section{Empirical Risk Minimization}
As stated above our goal is to minimize the generalization error, aka the
Vapnik risk as defined in \eqref{eq:vrisk}. However in general we do not have
access to the distribution over which the expctation is taken so that we
need to approximate it. We thus define the empirical risk minimization
which is just the average of the loss function over the training set.
\begin{equation}
    \hat L(\omega) \eqbydef \frac{1}{n}\sum_{i=1}^n l(\yi,h_{\omega}(\xi))
    + R(\omega) \label{eq:erm}
\end{equation} 

The term $R(\omega)$ is a regularizer for the parameter in the above (e.g.
$\vert \vert \cdot \vert \vert ^2$). While we have the information necessary to
compute the above it is rarely easy to miminize; the structured loss functions
(in the summation above) which naturally arise are not convex and/or continuous
in general. While there are techniques to deal with non continuous objective
functions the absence of convexity is a major problem in optimization. 

The solution is to replace the loss function by a \emph{surrogate loss
function} (or a contrast function in the statistics literature).  While there
are many design consideration for such a surrogate loss function the most
commonly required feature is convexity. Examples for such functions
are the structured hinge loss and the log-loss. Yann LeCun 
presents and discusses many of these functions in his paper on energy based
models\footnote{\url{http://yann.lecun.com/exdb/publis/pdf/lecun-06.pdf}}.\\

Substituting those new loss functions into the expression for ERM we 
get the objective function of our optimization problem:
\begin{equation}
    \hat{\mathcal L}(\omega) \eqbydef \frac{1}{n}\sum_{i=1}^n
    \mathcal L (\xi,\yi,\omega)
    + R(\omega) \label{eq:surrogate}
\end{equation}

\newpage
%The following is copy pasted from 2017 partially latexed scribes by
%Gabriel Huang
\section{Generative vs. Discriminative models.}

Prediction consists in learning a mapping $h_w: \X \longrightarrow \Y$. This
can be done at different levels of modelling and robustness. There is a
tradeoff: a generative model is more powerful --since it explains the data more
completely-- but it is less robust, whereas a simpler, discriminative model
explains less but is more robust.

Types of model from least to most discriminative:
\begin{itemize}
\item \textbf{joint-probability $p_w(x,y)$} -- generative model of distribution
    of $x$ and $y$ jointly. The fit to data is the likelihood, and learning is
    done by finding the parameters $w$ maximizing the likelihood (ML). One way
    to make predictions to take $h(x) = \argmax_y p_w(y|x)= \argmax_y
    p_w(x,y)$.  \item \textbf{conditional probability $p_w(y|x)$} --
    conditional generative
    model of the distribution of $y$ once $x$ is fixed. The fit to data is the
    conditional-likelihood, and learning is done by finding the parameters $w$
    maximizing the conditional-likelihood (MCL). One way to make predictions to
    take $h(x) = \argmax_y p_w(y|x)$.  \item \textbf{prediction mapping
    $y=h_w(x)$} -- discriminative model: instead of a model of $x$ or $y$,
    directly learn a mapping from $x$ to a good label $y$.  The model is
    learned by surrogate loss minimization, which uses information from an
    \textbf{loss function} or \textbf{error function} $l(y,y')$, which in turn
    defines the learning task. 
\end{itemize}

The terminology ``structured prediction" traditionally focuses on prediction
mappings (vs.\ generative modelling), and will be the approach we take in this
course. Loosely speaking, because pointwise estimators are more discriminative,
they are simpler which gives them more power to model more complex labels $y$.
Even though the learned prediction function is not the correct one, we can
still try to find the best one -- in terms of classification error -- among a
class of functions.


\newpage
\section{Examples of structured prediction problems}

\begin{example}[optical character recognition]
from an input image $x$ representing a word or sentence, predict the vector $y$
of corresponding letters where each dimension corresponds to successive
letters.  \end{example}

\begin{example}[word alignment]
Align sentences in two languages.
\par
  \centering
    \includegraphics[width=0.8\textwidth]{img/wordalign}
\end{example}

\begin{example}[machine translation]
translate a sentence from one language to the other.
  \centering
    \includegraphics[width=0.8\textwidth]{img/googletranslate}
Google translate interface.

\end{example}

\begin{example}[3D object recognition]
from a depth map $x$ predict an object map $y$ which maps each pixel location
to an object class.
  \centering
    \includegraphics[width=0.5\textwidth]{img/3d}
Image courtesy of Ben Taskar.
\end{example}

\begin{example}[protein folding]
from a sequence of amino-acids $x$ predict the 3D structure $y$ of the
corresponding protein.
  \centering
    \includegraphics[width=0.7\textwidth]{img/proteinfolding}
Image courtesy of Ben Taskar.
\end{example}

\begin{example}[text parsing]
from a sentence of words $x$ predict the parse tree(s) $y$ that generated~$x$.
  \centering
    \includegraphics[width=0.5\textwidth]{img/parsetree}
Image courtesy of Ron
Daniel%
\footnote{https://www.elsevier.com/connect/new-open-access-resource-will-support-text-mining-and-natural-language-processing}.
\end{example}



\end{document}
